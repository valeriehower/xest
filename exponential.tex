\documentclass{ximera}
 
\input{preamble.tex}
 
\title[Dig-In:]{Exponential functions}
 
 
\begin{document}
\begin{abstract}
This section is adapted from the source code {\bf calculus/reviewOfFamousFunctions/digInExponentialAndLogarithmeticFunctions.tex} available at \url{https://github.com/mooculus/calculus}
\end{abstract}
\maketitle
 


\section{What are exponential functions?}
 

\begin{definition}
  An \dfn{exponential function} is a function of the form
  \[
  f(x) = a^x
  \]
  where  $a\ne 1$ is a positive real number. The domain of an
  exponential function is $(-\infty,\infty)$.
\end{definition}

\begin{question}
  Is $4^{-x}$ an exponential function?
  \begin{prompt}
  \begin{multipleChoice}
    \choice[correct]{yes}
    \choice{no}
  \end{multipleChoice}
  \end{prompt}
  \begin{feedback}
    Note that
    \[
    4^{-x} = \left(4^{-1}\right)^x = \left(\frac{1}{4}\right)^x.
    \]
  \end{feedback}
\end{question}



 
\section{What can the graphs look like?}
 
\subsection{Graphs of exponential functions}
 
\begin{example}
  Here we see the the graphs of four exponential functions.
  \begin{image}
    \begin{tikzpicture}
      \begin{axis}[
          domain=-2:2,
          xmin=-2, xmax=2,
          ymin=-.5, ymax=4,
          axis lines =middle, xlabel=$x$, ylabel=$y$,
          every axis y label/.style={at=(current axis.above origin),anchor=south},
          every axis x label/.style={at=(current axis.right of origin),anchor=west},
        ]
	\addplot [very thick, penColor, smooth] {e^x};
        \addplot [very thick, penColor2, smooth] {2^x)};
        \addplot [very thick, penColor3, smooth] {(1/2)^x)};
        \addplot [very thick, penColor4, smooth] {(1/3)^x)};
        
        
        
        \node at (axis cs:-1.5, 2 ) [penColor3,anchor=west] {$A$};
        \node at (axis cs:-.8, 2.6 ) [penColor4,anchor=west] {$B$};
        \node at (axis cs:0.6, 2.6 ) [penColor,anchor=west] {$C$};
        \node at (axis cs:1.2, 2 ) [penColor2,anchor=west] {$D$};
        
      \end{axis}
    \end{tikzpicture}
  \end{image}
  Match the curves $A$, $B$, $C$, and $D$ with the functions
  \[
  e^x, \qquad \left(\frac{1}{2}\right)^{x}, \qquad  \left(\frac{1}{3}\right)^{x}, \qquad 2^{x}.
  \]
  \begin{explanation}
    One way to solve these problems is to compare these functions
    along the vertical line $x=1$,
    \begin{image}
      \begin{tikzpicture}
        \begin{axis}[
            domain=-2:2,
            xmin=-2, xmax=2,
            ymin=-.5, ymax=4,
            axis lines =middle, xlabel=$x$, ylabel=$y$,
            every axis y label/.style={at=(current axis.above origin),anchor=south},
            every axis x label/.style={at=(current axis.right of origin),anchor=west},
          ]
	  \addplot [very thick, penColor, smooth] {e^x}; %C
          \addplot [very thick, penColor2, smooth] {2^x)};%D
          \addplot [very thick, penColor3, smooth] {(1/2)^x)};%A
          \addplot [very thick, penColor4, smooth] {(1/3)^x)};%B
            
          \node at (axis cs:-1.5, 2 ) [penColor3,anchor=west] {$A$};
          \node at (axis cs:-.8, 2.6 ) [penColor4,anchor=west] {$B$};
          \node at (axis cs:0.6, 2.6 ) [penColor,anchor=west] {$C$};
          \node at (axis cs:1.2, 2 ) [penColor2,anchor=west] {$D$};

          \addplot [textColor, dashed] plot coordinates {(1,-.5) (1,4)};

          \addplot[color=penColor,fill=penColor,only marks,mark=*] coordinates{(1,e)}; %C
          \addplot[color=penColor2,fill=penColor2,only marks,mark=*] coordinates{(1,2)}; %D
          \addplot[color=penColor3,fill=penColor3,only marks,mark=*] coordinates{(1,1/2)}; %A
          \addplot[color=penColor4,fill=penColor4,only marks,mark=*] coordinates{(1,1/3)}; %B
        \end{axis}
      \end{tikzpicture}
    \end{image}
    Note
    \[
    \left(\frac{1}{3}\right)^1 < \left(\frac{1}{2}\right)^1  < 2^1 < e^1.
    \]
    Hence we see:
    \begin{itemize}
    \item $\left(\frac{1}{3}\right)^{x}$ corresponds to
      $\answer[given]{B}$.
    \item $\left(\frac{1}{2}\right)^{x}$ corresponds to $\answer[given]{A}$.
    \item $2^x$ corresponds to $\answer[given]{D}$.
    \item $e^x$ corresponds to $\answer[given]{C}$.
    \end{itemize}
  \end{explanation}
\end{example}

 
\section{Properties of exponential functions}
 
Working with exponential functions is often simplified by 
applying properties of these functions.  We will use these properties throughout the semester.
 
\subsection{Properties of exponents}
Let $a$ be a positive real number with $a\ne 1$.
\begin{itemize}
  \item $a^m\cdot a^n = a^{m+n}$
  \item $\left(a^m\right)^n = a^{mn}$
  \item $a^{-1} = \frac{1}{a}$
\end{itemize}
\begin{question}
  What exponent makes the following true?
  \[
  2^4 \cdot 2^3 = 2^{\answer{7}}
  \]
  \begin{hint}
    \[
    (2^4) \cdot (2^3) = (2 \cdot 2\cdot 2 \cdot 2) \cdot  (2 \cdot 2\cdot 2)
    \]
  \end{hint}
\end{question}

 \begin{example}
Solve the exponential equation.
\[
27^x=9
\]
\end{example}
 \begin{explanation}
Once we have the same base on both sides of the equation, we may equate exponents.  In this example, we observe that $27$ and $9$ are both powers of $3$.  That is, $27=3^{\answer[given]{3}}$ and $9=3^{\answer[given]{2}}$, and therefore we can write the exponential equation as 
 \[
3^{\answer[given]{3x}}=3^{\answer[given]{2}}.
\]
By equating exponents, we arrive at $x=\frac{2}{3}$.
\end{explanation}
 
 
\end{document}