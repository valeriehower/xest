\documentclass{ximera}
\input{preamble.tex}
 
\title[Dig-In:]{The Chain Rule}



\begin{document}
\begin{abstract}
The Chain Rule.  This section is adapted from Applied Calculus, by Shana Calaway, Dale Hoffman, and David Lippman, used under the Creative Commons Attribution license
\end{abstract}
\maketitle

The  Chain  Rule  will  let  us  find  the  derivative  of  a  composition.  
\begin{example}
Find the derivative of $y=(4x^3+15x)^2$

This is not a simple polynomial, so we can’t use the power rule yet.  This function is a product, so we could write it as $(4x^3+15x)(4x^3+15x)$ and use the product rule.  Or we could multiply/F.O.I.L it out and simply differentiate the resulting polynomial.  Let's do the second way:
 \[ 
y=(4x^3+15x)(4x^3+15x) =16x^6+120x^4+225x^2 
\]
\[
y^{\prime}=64x^5+480x^3+450x
\]
\end{example}
Now suppose we want to find the derivative of $y=(4x^3+15x)^{20}$.  We \textbf{could} write it as a product with $20$ factors and use the product rule, or we \textbf{could} multiply it out.  But I don't want to do that, do you?

We need an easier way, a rule that will handle a composition like this.  The Chain Rule is a little complicated, but it saves us the much more complicated algebra of multiplying something like this out.  It will also handle compositions where it wouldn't be possible to ``multiply it out.''

The Chain Rule is the most common place for students to make mistakes.  Part of the reason is that the notation takes a little getting used to.  And part of the reason is that students often forget to use it when they should.  When should you use the Chain Rule?  Almost every time you take a derivative.

\begin{formula}[Chain Rule]
Here $f(x)$ and $g(x)$ are differentiable functions and $y=f(g(x))$.
$$ \frac{d}{dx} f(g(x)) = f^{\prime}(g(x))\cdot g^{\prime}(x)$$
in words: The derivative of a composition is the derivative of the outside, with the inside staying the same, TIMES the derivative of the inside function.
  
\end{formula}
\begin{example}
Find the derivative of $y=(4x^3+15x)^2$.

This is the same one we did before by multiplying out.  This time, let's use the Chain Rule:  The inside function is what appears inside the parentheses: $g(x)=\answer[given]{4x^3+15x}$.  We imagine replacing the entire inside function with an ``$x$'' and can read off the outside function $f(x)=\answer[given]{x^2}$.

To use the chain rule, we need the derivatives of both the inside and outside functions.  
\[
\boxed{
\begin{array}{ll}
f(x)=x^2 & f^{\prime}(x)=2x \\
g(x)= 4x^3+15x& g^{\prime}(x)=12x^2+15 
\end{array}
}
\]
We use the chain rule:
\[
 y^{\prime}=f^{\prime}(g(x))\cdot g^{\prime}(x) =2(4x^3+15x)\cdot(12x^2+15)
\]
If you multiply this out, you get the same answer we got before.  Hurray!  Algebra works!
\end{example}
\end{document}
