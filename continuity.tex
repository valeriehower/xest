\documentclass{ximera}
 
\input{preamble.tex}
 
\outcome{Identify continuous functions from their graphs.}
\outcome{Identify $x$-values where a function is discontinuous from a graph.}

\title[Dig-In:]{Continuity}
\begin{document}
\begin{abstract}
The idea of continuity.
\end{abstract}
\maketitle
 

 
\begin{idea}
  A function $f$ is \dfn{continuous at} $x=a$ if you can trace through the point $(a,f(a))$ without lifting your pen.

 \youtube{hXFLvVFQa5k}
\end{idea}
\begin{question}
  Consider the graph of  the function $f$
  \begin{image}
    \begin{tikzpicture}
      \begin{axis}
    [xmin=-0.2,
          xmax=2.2,
          ymin=-0.2,
          ymax=2.2,
          axis lines=center,
          xlabel=$x$,ylabel=$y$,
          every axis y label/.style={at=(current axis.above origin),anchor=south},
          every axis x label/.style={at=(current axis.right of origin),anchor=west},
      domain=-1:2,
          clip=false,
      ytick={0.5,1,1.5,2},
      yticklabels={$0.5$,$1$,$1.5$,$2$},
      xtick={0.5,1.0,1.5,2},
      xticklabels={$0.5$,$1$,$1.5$,$2$},
      grid = major
    ]
        \addplot[very thick,penColor] plot coordinates {(0,0) (1,1)};
        \addplot[very thick,penColor] plot coordinates {(1,2) (2,0)};
         
        %\draw[very thin,color=black] (axis cs:0,-0.2) -- (axis cs:0,2);
 
        \addplot[color=penColor,fill=background,only marks,mark=*] coordinates{(1,1)};  %% open hole
        \addplot[color=penColor,fill=background,only marks,mark=*] coordinates{(.5,.5)};  %% open hole
 
        \addplot[color=penColor,fill=penColor,only marks,mark=*] coordinates{(0,0)};  %% closed hole
        \addplot[color=penColor,fill=penColor,only marks,mark=*] coordinates{(2,0)};  %% closed hole
        \addplot[color=penColor,fill=penColor,only marks,mark=*] coordinates{(1,2)};  %% closed hole
         
        %% \draw[fill=black] (axis cs:0,0) circle [radius=2pt];
    %% \draw[fill=black] (axis cs:2,0) circle [radius=2pt];
        %% \draw[fill=black] (axis cs:1,2) circle [radius=2pt];
     
    \end{axis}
    \end{tikzpicture}
  \end{image}
  Which of the following are true?
  \begin{multipleChoice}
    \choice{$f$ is continuous at $x=0.5$}
    \choice{$f$ is continuous at $x=1$}
    \choice[correct]{$f$ is continuous at $x=1.5$}
  \end{multipleChoice}
   
\end{question}
 
 
\begin{example}
Give $x$-values where the function below is discontinous (i.e. not
continuous).
\begin{image}
\begin{tikzpicture}
    \begin{axis}[
            domain=0:10,
            ymax=5,
            ymin=0,
            %samples=100,
            axis lines =middle, xlabel=$x$, ylabel=$y$,
            every axis y label/.style={at=(current axis.above origin),anchor=south},
            every axis x label/.style={at=(current axis.right of origin),anchor=west},
            %% ytick={0.5,1,1.5,2},
        %% yticklabels={$0.5$,$1$,$1.5$,$2$},
        %% xtick={0.5,1.0,1.5,2},
        %% xticklabels={$0.5$,$1$,$1.5$,$2$},
        grid = major
          ]
      \addplot [very thick, penColor, smooth, domain=(4:10)] {3 + sin(deg(x*2))/(x-1)};
          \addplot [very thick, penColor, smooth, domain=(0:4)] {1};
          \addplot[color=penColor,fill=background,only marks,mark=*] coordinates{(4,3.30)};  %% open hole
          \addplot[color=penColor,fill=background,only marks,mark=*] coordinates{(2,1)};  %% open hole
          \addplot[color=penColor,fill=background,only marks,mark=*] coordinates{(6,2.893)};  %% open hole
          \addplot[color=penColor,fill=penColor,only marks,mark=*] coordinates{(4,1)};  %% closed hole
          \addplot[color=penColor,fill=penColor,only marks,mark=*] coordinates{(6,2)};  %% closed hole
 
          \addplot[color=penColor,fill=penColor,only marks,mark=*] coordinates{(0,1)};  %% closed hole
          \addplot[color=penColor,fill=penColor,only marks,mark=*] coordinates{(10,3.1)};  %% closed hole
        \end{axis}
\end{tikzpicture}
%% \caption{A plot of a function with discontinuities at $x=4$ and $x=6$.}
%% \label{plot:discontinuous-function}
\end{image}
 
\begin{explanation}
  To start, $f$ is not even defined at $x=\answer[given]{2}$, therefore $f$
  cannot be continuous at $x=\answer[given]{2}$ as you must lift your pen over the hole.
 
  Next, from the plot above we see that $\lim_{x\to 4} f(x)$ does not
  exist because
  \[
  \lim_{x\to 4^-}f(x) = \answer[given]{1}\qquad\text{and}\qquad \lim_{x\to 4^+}f(x) \approx 3.3
  \]
  This causes a ``jump'' in the function at $x=4$ where you must lift your pen. This means $f$ cannot be continuous
  at $x=4$.
 
  We also see that $\lim_{x\to 6} f(x)$ exists and $\lim_{x\to 6} f(x)\approx 2.9$ .  However, as $f(6) =
  \answer[given]{2}$, we must pick up our pen to color in the point $(6,f(6))$ as we trace along the graph. This means $f$
  is not continuous at $x=6$.
\end{explanation}
\end{example}
 
 
\begin{remark}
The following common functions are continuous at every $x$-value in their domains.  Try visualizing some of their graphs to convince yourself of the continuity. 
\begin{description}
\item[Constant function]\index{constant function} $f(x) =k$ for $k$ a real number
\item[Power functions]\index{power function} $f(x)=x, f(x)=x^2, f(x)=x^3, \cdots$ or more generally $f(x)=x^r$ for $r>0$
\item[Exponential function]\index{exponential function} $f(x)=a^x$ with base $a>0$
\item[Logarithmic function]\index{logarithmic function} $f(x)=\log_a(x)$ with base $a>0$
\end{description}

 
\end{remark}
 
 
 

 
\end{document}
